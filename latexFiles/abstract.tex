\chapter*{ Abstract}
\selectlanguage{english}
In this work we are interested in a way to describe discrete
Markov chains, whose transition matrices have positive determinant, in function of continuous Markov chains and vice versa.
Such procedure is possible if we study the algebraic properties of
the group of Lie matrices given by the stochastic matrices and its
tangent space, which satisfy the condition of being a Lie algebra.
Also, if we took the coproduct of an elemente in the center of 
the tangent space, just like the Casimir
element, we obtain a generator matrix of a continuous Markov process. And,
given this generator matrix, we are capable to define a particle system
of two sites, in which, under some algebraic tranformations,
we can extand to a particle system of N sites. Finnaly,
we construct a self-duality operator to a particle system of N sites ($N \geq 2$)
and we proove the stochastic version of Noether's Theorem.
%Surprisingly, this algebraic properties give to us a lot of information
%about continuous and discrete Markov chains.
%Em posse desta matriz geradora somos capazes de construir um sistema de duas partículas, o
%qual, sob certas transformações, podemos estender para um sistema de N-partículas. Finalmente,
%obtemos operadores de autodualidade para sistema de N-partículas ($N \geq 2$) e enunciamos
%a versão estocástica do Teorema de Noether.


\vspace{.5cm}
\textbf{Keywords}:\keywords

\selectlanguage{portuguese}
