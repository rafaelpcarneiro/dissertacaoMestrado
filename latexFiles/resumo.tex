\chapter*{Resumo}
Neste trabalho estamos interessados em descrever as cadeias de Markov a tempo discreto, cujas
matrizes de transição tenham determinante positivo, em função
de processos markovianos a tempo contínuo, e vice-versa. Tal procedimento é possível se estudarmos 
as propriedades algébricas do grupo de matrizes de Lie, dado pelas matrizes estocásticas, e as propriedades 
algébricas de seu espaço
tangente, o qual será uma álgebra de Lie. 
Ainda mais, ao estudarmos o coproduto
de um elemento do centro do espaço tangente, como o Casimir, obtemos uma forma
de construir uma matriz geradora de um processo de Markov a tempo contínuo.
E, em posse desta matriz geradora, somos capazes de estabelecer um sistema de partículas de dois sítios, o
qual, sob certas transformações, podemos estender para sistemas de N-partículas. Para encerrar,
obtemos operadores de autodualidade para sistemas de N-partículas ($N \geq 2$) e provamos
a versão estocástica do Teorema de Noether.
%De forma surpreendente, estas propriedades
%algébricas nos trazem muita informação sobre as cadeias de Markov a tempo contínuo e a tempo discreto.

\vspace{.5cm}
\textbf{Palavras-chave}:\palavraschaves



